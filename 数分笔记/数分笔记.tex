\documentclass[12pt]{ctexart}
\usepackage{amsmath, amsthm, amssymb, graphicx}
\usepackage{bm}%允许在数学模式中输入粗体,命令\bm{}
\usepackage{mathrsfs}%字体宏包
\usepackage{enumerate}%列表宏包
\usepackage{geometry}
\usepackage{color}
\usepackage[hidelinks,bookmarks=true,colorlinks=true,linkcolor=blue,bookmarksnumbered=true]{hyperref}
\title{数学分析笔记}%标题
\author{关茂松}%名字
\pagestyle{plain}%页眉为空,页脚为页码
\geometry{top=25.4mm,bottom=25.4mm,left=20mm,right=20mm,headheight=2.17cm,headsep=4mm,footskip=12mm}
\theoremstyle{definition}%定理格式为粗体标签,正体内容
\newtheorem{lemma}{引理}[subsection]%引理环境,命令\begin{lemma} \end{lemma}
\newtheorem{theorem}{定理}[subsection]%定理环境,命令\begin{theorem} \end{theorem}
\newtheorem{corollary}{推论}[subsection]%推论环境, 命令\begin{corollary} \end{corollary}
\newtheorem{definition}{定义}[subsection]%定义环境,命令\begin{definition} \end{definition}
\newtheorem{proposition}{命题}[subsection]%命题环境,命令\begin{proposition} \end{proposition}
\newtheorem{example}{例子}[subsection]%例子环境,命令\begin{example} \end{example}
\newtheorem*{remark}{注}
\newcommand{\scr}[1]{\mathscr{#1}}%自定义命令,花体拉丁字母,命令\scr{}
\newcommand{\rmnum}[1]{\romannumeral #1}%自定义命令,\rmnum{}用于输入小写罗马数字
\newcommand{\Rmnum}[1]{\uppercase\expandafter{\romannumeral #1}}%自定义命令,\Rmnum{}用于输入大写罗马数字
\newcommand{\dif}{\mathrm{d}}%自定义命令,方便输入微分符号
\newcommand{\dis}{\displaystyle}%自定义命令,方便使用行间公式尺寸
\date{}
\begin{document}
\maketitle
\tableofcontents%若要生成目录,则将这行前面的代码取消注释
%下面是正文部分
\section{实数与数列极限}
	\subsection{实数的基本性质与常用不等式}
	无.
	\subsection{数列与数列极限的概念}
	无.
	\subsection{收敛数列的性质}
	无.
	\subsection{发散数列与子列的概念}	
	无.
	\subsection{确界原理}
	无.
	\subsection{数列收敛的判别法}
	\begin{theorem}[压缩映像原理]
		$\ $
		\begin{enumerate}
			\item 对于任一数列$\{x_n\}$而言,若存在常数$r$,使得$\forall n\in \mathbb{N}$,恒有
			\[|x_{n+1}-x_n|\leqslant r|x_n-x_{n-1}|,0<r<1.\]
			则数列$\{x_n\}$收敛.
			\item 
			特别,若数列$\{x_n\}$利用递推公式给出:$x_{n+1}=f(x_n)\quad(n=1,2,\cdots)$,其中$f$为某一可微函数,且$\exists r\in \mathbb{R}$,使得
			\[|f^\prime(x)|\leqslant r<1,\]
			则数列$\{x_n\}$收敛.
		\end{enumerate}
	\end{theorem}
	\begin{remark}
		压缩映像条件是充分的而不是必要的.
	\end{remark}
	\begin{theorem}[不动点方法]
		已知数列$\{x_n\}$在区间$I$上由$x_{n+1}=f(x_n)\quad (n=1,2,\cdots)$给出,$f$是$I$上连续增函数,若$f$在$I$上有不动点$x^*$(即$x^*=f(x^*)$)满足
		\begin{equation}\label{4}
			(x_1-f(x_1))(x_1-x^*)\geqslant 0,
		\end{equation}
		则此时数列列$\{x_n\}$必收敛,且极限$A$满足$A=f(A)$.
		
		若\ref{4}式"$\geqslant$"改为"$>$"对任意$x_1\in I$成立,则意味着$x^*$是唯一不动点,并且$A=x^*$.
		
		\textbf{特别},若$f$可导,且$0<f^\prime(x)<1\quad (x\in I)$,则$f$严增,且不等式\ref{4}($\geqslant$"可改为"$>$)会自动满足($\forall x_1 \in I$).这时$f$的不动点存在且唯一,从而$A=x^*$.
	\end{theorem}
	设$z=f(x,y)$为二元函数,我们约定,将$z=f(x,x)$的不动点称为$f$的不动点(或二元不动点).
	\begin{theorem}[不动点方法的推广]
		已知$z=f(x,y)$为$x>0,y>0$上定义的正连续函数,$z$分别对$x$和$y$单调递增,假若:
		\begin{enumerate}
			\item 存在点$b$是$f(x,y)$的不动点;
			\item 当且仅当$x>b$时有$x>f(x,x)$.
		\end{enumerate}
		令$a_1=f(a,a),a_2=f(a_1,a)\enspace (a>0)$,
		\[a_n=f(a_{n-1},a_{n-2})\quad ,n=3,4,\cdots,\]
		则$\{a_n\}$单调有界,有极限,且其极限$A$是$f$的不动点.
	\end{theorem}
	\begin{remark}
		$\ $
		\begin{enumerate}
			\item 当$a_1\geqslant a$时,$\{a_n\}$单调递增有上界$b$,当$a_1<a$时,$\{a_n\}$单调递减有下界$b$,
			\item 按$b$的条件可知$b$是$f$的最大不动点,$x>b$时不可能再有不动点,
			\item 当$a_1<a$时,极限$A\geqslant b$是不动点,表明此时$A=b$.
		\end{enumerate}
	\end{remark}
	\subsection{Stolz公式}
	\begin{theorem}[$\dfrac{\infty}{\infty}$型Stolz公式]
		设$\{x_n\}$严格递增(即$\forall n\in \mathbb{N}$有$x_n<x_{n+1}$,且$\lim\limits_{n\to \infty}x_n=+\infty.$
		\begin{enumerate}
			\item 若$\lim\limits_{n\to \infty}\dfrac{y_n-y_{n-1}}{x_n-x_{n-1}}=a\text{(有限数)}$,则$\lim\limits_{n\to\infty}\dfrac{y_n}{x_n}=a$;
			\item 若$\lim\limits_{n\to \infty}\dfrac{y_n-y_{n-1}}{x_n-x_{n-1}}=+\infty$,则$\lim\limits_{n\to\infty}\dfrac{y_n}{x_n}=+\infty$;
			\item 若$\lim\limits_{n\to \infty}\dfrac{y_n-y_{n-1}}{x_n-x_{n-1}}=-\infty$,则$\lim\limits_{n\to\infty}\dfrac{y_n}{x_n}=-\infty$;
		\end{enumerate}
	\end{theorem}
\section{函数与函数极限}
	\subsection{映射与函数的概念}
	无.
	\subsection{\texorpdfstring{$x\rightarrow\infty$}时函数极限的概念}
	无.
	\subsection{\texorpdfstring{$x\rightarrow x_0$}时函数极限的概念}
	无.
	\subsection{函数极限的性质}
	无.
	\subsection{函数极限存在的判别法}
	\begin{theorem}[归结原则]
		设函数$f(x)$在$\mathring{U}(x_0;\delta_0)$内有定义,$A\in \mathbb{R}$,则$\lim\limits_{x\rightarrow x_0}f(x)=A$的充要条件是:对$\mathring{U}(x_0;\delta_0)$内任何以$x_0$为极限的数列$\{x_n\}$,都有
		\[\lim_{n\rightarrow \infty}f(x_n)=A.\]
	\end{theorem}
	\begin{theorem}
		设函数$f(x)$在$U(\infty)=\{x||x|>M\geqslant0\}$内有定义.如果在$\mathring{U}(x_0;\delta_0)$内有定义,$A\in \mathbb{R}$,则$\lim\limits_{x\rightarrow\infty}f(x)=A$的充要条件是:对$U(\infty)$内任何以$\infty$为非正常极限的数列$\{x_n\}$,都有$\lim\limits_{n\rightarrow\infty}f(x_n)=A.$
	\end{theorem}
\section{函数的连续性}
	\subsection{连续函数的概念}
	\begin{proposition}
		设$f(x)$在有限开区间$(a,b)$内连续,则$f(x)$在$(a,b)$内一致连续的充要条件是极限$\dis \lim_{x\to a^+}f(x)$及$\dis \lim_{x\to b^-}f(x)$存在(有限).
	\end{proposition}
	\begin{proposition}
		若$f(x)$在$[a,+\infty]$上连续,$\dis \lim_{x\to +\infty}f(x)=A$(有限),则$f(x)$在$[a,+\infty)$一致连续.
	\end{proposition}
	\begin{proposition}
		设$f(x)$在$[a,+\infty)$上一致连续,$\varphi(x)$在$[a,+\infty)$上连续,$\dis\lim_{x\to +\infty}[f(x)-\varphi(x)]=0$,则$\varphi(x)$在$[a,+\infty)$上一致连续.
	\end{proposition}
	\subsection{函数间断的概念}
	无.
	\subsection{连续函数的局部性质与初等函数的连续性}
	无.
	\subsection{连续函数的基本性质}
	无.
\section{微分与导数}
	\subsection{微分与导数的概念}
	无.
	\subsection{求导方法与导数公式}
	无.
	\subsection{微分的计算与应用}
	无.
	\subsection{高阶导数与高阶微分}
	无.
	\subsection{参数方程所表示的函数的导数}
	\begin{definition}\label{1}
		设函数$\varphi(t),\psi(t)$在区间$[\alpha,\beta]$上有连续的导数,且
		\[\left[\varphi^\prime(t)\right]^2+\left[\psi^\prime(t)\right]^2\not=0,\]
		这时称方程
		\[
		\begin{cases}
			x=\varphi(t),\\
			y=\psi(t),
		\end{cases}
		t\in[\alpha,\beta]
		\]
		给出的曲线为光滑曲线.
	\end{definition}
\section{导数的应用}
	\subsection{Fermat定理和Darboux定理}
	\begin{theorem}[Fermat定理]
		若$f(X)$在点$x_0$可导,且$x_0$是$f(x)$的极值点,则$f^\prime(x_0)=0$.
	\end{theorem}
	\begin{theorem}[Darboux导函数介值定理]
		设函数$f(x)$在$[a,b]$上可导,且$f^\prime_+(a)\not=f^\prime
		_-(b)$,$\eta$介于$f^\prime_+(a)$与$f^\prime_-(b)$之间,则存在$\xi\in(a,b)$,使得$f^\prime(\xi)=\eta$.
	\end{theorem}
	\subsection{中值定理}
	\begin{proposition}[推广的柯西中值定理]
		设$(a,b)$为有限或无穷区间,$f(x)$在$(a,b)$内可微,且$\dis\lim_{x\to a^+}f(x)=\lim_{x\to b^-}f(x)=A$(有限或$\pm\infty$),则$\exists \xi\in(a,b)$,使得$f^\prime(\xi)=0$.
	\end{proposition}
	\begin{proof}
		若$f(x)\equiv A$,则显然;否则先使用连续函数的介值性,然后再使用罗尔中值定理.
	\end{proof}
	\begin{proposition}
		设$f(x)$在$[a,b]$上连续,且$(a,b)$内可微,若存在极限$\dis\lim_{x\to a^+}f^\prime(x)=l$,则右导数$f_+^\prime(a)$存在且等于$l$.
	\end{proposition}
	\subsection{不定式极限}
	无.
	\subsection{Taylor公式}
	无.
	\subsection{函数的单调性与凸性}
	设函数$f(x)$在区间$I$上有定义,$f(x)$在$I$上称为凸函数,当且仅当$\forall x_1,x_2\in I,\forall\lambda\in(0,1)$,有
	\begin{equation}\label{eq2}
		f\left( \lambda x_1+\left( 1-\lambda \right) x_2 \right) \leqslant \lambda f\left( x_1 \right) +\left( 1-\lambda \right) f\left( x_2 \right) .
	\end{equation}
	若式\eqref{eq2}中的"$\leqslant$"改成"<",则是严格凸函数的定义;若"$\leqslant$"改成"$\geqslant$"或"$>$",则分别是凹函数与严格凹函数的定义.
	\begin{theorem}
		设$f(x)$在区间$I$上有定义,则以下条件等价(其中各不等式要求$\forall x_1,x_2,x_3\in I,x_1<x_2<x_3$保持成立):
		\begin{enumerate}
			\item $f(x)$在I上是凸函数;
			\item $\dfrac{f\left( x_2 \right) -f\left( x_1 \right)}{x_2-x_1}\leqslant \dfrac{f\left( x_3 \right) -f\left( x_1 \right)}{x_3-x_1}$;
			\item $\dfrac{f\left( x_3 \right) -f\left( x_1 \right)}{x_3-x_1}\leqslant \dfrac{f\left( x_3 \right) -f\left( x_2 \right)}{x_3-x_2}$;
			\item $\dfrac{f\left( x_2 \right) -f\left( x_1 \right)}{x_2-x_1}\leqslant \dfrac{f\left( x_3 \right) -f\left( x_2 \right)}{x_3-x_2}$
			\item 曲线$y=f(x)$上三点$A(x_1,f(x_1)),B(x_2,f(x_2)),C(x_3,f(x_3))$所围成的有向图面积
			$$\dfrac{1}{2}\left| \begin{matrix}
				1&		x_1&		f\left( x_1 \right)\\
				1&		x_2&		f\left( x_2 \right)\\
				1&		x_3&		f\left( x_3 \right)\\
			\end{matrix} \right|\geqslant 0$$.
		\end{enumerate}
	\end{theorem}
	\begin{theorem}
		若$f(x)$是区间$I$上的凸函数,则对$I$上任一内点$x$,单侧导数$f_+^\prime(x),f_-^\prime(x)$皆存在,皆为增函数,且
		\[f_-^\prime(x)\leqslant f_+^\prime(x)\quad(\forall x\in I^\mathrm{o}).\]
		这里$I^\mathrm{o}$表示$I$的全体内点组成之集合(若$f$为严格凸的,则$f_+^\prime(x)$与$f_-^\prime(x)$为严格递增的).
	\end{theorem}
	\begin{corollary}
		若$f(x)$在区间$I$上为凸的,则$f$在任一内点$x\in I^\mathrm{o}$上连续.
	\end{corollary}
	\begin{theorem}
		设$f(x)$在区间$I$上有定义,则$f(x)$为凸函数的充要条件是:$\forall x_0\in I^\mathrm{o}$,$\exists$实数$\alpha$,使得$\forall x\in I$有$f(x)\geqslant\alpha (x-x_0)+f(x_0)$.
	\end{theorem}
	\begin{theorem}
		设$f(x)$在区间$I$上有导数,则$f(x)$在$I$上为凸函数的充要条件是$f^\prime(x)$单调递增($x\in I$).
	\end{theorem}
	\begin{corollary}
		设$f(x)$在区间$I$上有二阶导数,则$f(x)$在区间$I$上为凸函数的充要条件是$f^{\prime\prime}(x)\geqslant 0$.
	\end{corollary}
	\subsection{函数的极值与最值}
	无.
	\subsection{函数作图}
	\begin{theorem}
		直线$y=kx+b$是曲线$y=f(x)$的斜渐近线等价于
		\[k=\lim_{x\rightarrow+\infty}\frac{f(x)}{x},b=\lim_{x\rightarrow+\infty}[f(x)-kx].\]
	\end{theorem}
\section{实数集的稠密性与完备性}
	\subsection{实数集的稠密性}
	无.
	\subsection{实数集的完备性}
	无.
	\subsection{上极限和下极限简介}
	无.
\section{不定积分}
	\subsection{原函数与不定积分的概念}
	无.
	\subsection{不定积分的计算}
	无.
	\subsection{有理函数的不定积分}
	无.
\section{定积分}
	\subsection{定积分的概念与性质}
	\begin{theorem}[可积的必要条件]
		设函数$f(x)$在$[a,b]$上可积,则$f(x)$在$[a,b]$上有界.
	\end{theorem}
	\subsection{微积分基本定理}
	\begin{theorem}
		设$f(x)$在$[a,b]$上可积,则变上限的定积分所定义的函数
		\[F(x)=\int_a^xf(t)\dif t,\quad x\in [a,b]\]
		在$[a,b]$上连续.
	\end{theorem}
	\begin{theorem}[原函数存在定理]
		设$f(x)$在$[a,b]$上连续,则变上限的定积分所定义的函数$F(x)$是$f(x)$在$[a,b]$上的原函数.
	\end{theorem}
	\begin{theorem}[微积分基本定理]
		设$f(x)$在$[a,b]$上连续,$F(x)$是$f(x)$在$[a,b]$上的任一原函数, 则
		\[\int_{a}^{b}f(t)\dif t=F(b)-F(a).\]
	\end{theorem}
	\subsection{定积分的计算}
	无.
	\subsection{定积分存在的条件}
	\begin{definition}[达布和]
		设$f(x)$在$[a,b]$上有界,$T:a=x_0<x_1<x_2<\cdots<x_{n-1}<x_n=b$为$[a,b]$上的任一分割,记$\Delta_i=[x_{i-1},x_i](i=1,2,\cdots,n)$,则$f(x)$在$[a,b]$及每个$\Delta_i$上都存在上、下确界.记
		\[M=\sup_{x\in[a,b]}f(x),\quad m=\inf_{x\in[a,b]}f(x),\]
		\[M_i=\sup_{x\in\Delta_i}f(x),\quad m_i=\inf_{x\in\Delta_i}f(x),\quad i=1,2,\cdots,n.\]
		作和
		\[S(T)=\sum_{i=1}^{n}M_i\Delta x_i,\quad s(T)=\sum_{i=1}^{n}m_i\Delta x_i.\]
		分别称为$f$关于分割$T$的上和与下和(或称为达布上和与达布下和,统称为达布和).
	\end{definition}
	下面的定理\ref{theo8.4.1}到定理\ref{theo8.4.4}都是对$[a,b]$上的有界函数$f(x)$来说的.
	\begin{theorem}\label{theo8.4.1}
		对$[a,b]$的同一分割$T$,相对于任何点集$\{\xi_i\}$而言,上和是所有积分和的上确界,下和是所有积分和的下确界,即
		$$S\left( T \right) =\mathop {\mathrm{sup}} \limits_{\left\{ \xi _i \right\}}\sum_{i=1}^n{f\left( \xi _i \right) \Delta x_i},\quad s\left( T \right) =\mathop {\mathrm{inf}} \limits_{\left\{ \xi _i \right\}}\sum_{i=1}^n{f\left( \xi _i \right) \Delta x_i}.$$
	\end{theorem}
	\begin{theorem}
		设$T^\prime$为分割$T$添加$p$个新分点所得到的分割,则
		$$s\left( T \right) \leqslant s\left( T^{\prime} \right) \leqslant s\left( T \right) +p\left( M-m \right) \parallel T\parallel ,$$
		$$S\left( T \right) \geqslant S\left( T^{\prime} \right) \geqslant S\left( T \right) -p\left( M-m \right) \parallel T\parallel ,$$
		即分点增加后,下和递增,上和递减.
	\end{theorem}
	\begin{theorem}
		若$T^\prime$与$T^{\prime\prime}$为$[a,b]$的任意两个分割,$T$为$T^\prime$与$T^{\prime\prime}$的所有分点合并后得到的分割(注意:重复的分点只取一次),记为$T=T^\prime+T^{\prime\prime}$,则
		$$S\left( T \right) \leqslant S\left( T^{\prime} \right) ,\quad s\left( T \right) \geqslant s\left( T^{\prime} \right) ,$$
		$$S\left( T \right) \leqslant S\left( T^{\prime \prime} \right) ,\quad s\left( T \right) \geqslant s\left( T^{\prime \prime} \right) .$$
	\end{theorem}
	\begin{theorem}\label{theo8.4.4}
		对$[a,b]$的任意两个分割$T^\prime$与$T^{\prime\prime}$总有
		$$s\left( T^{\prime} \right) \leqslant S\left( T^{\prime \prime} \right) ,\quad s\left( T^{\prime \prime} \right) \leqslant S\left( T^{\prime} \right) ,$$
		即下和不超过上和.
	\end{theorem}
	因此,对所有分割来说,所有的下和有上界,所有的上和有下界,从而分别有上、下确界,记作$s$与$S$,即
	\[s=\sup_{T}\{s(T)\},\quad S=\inf_{T}\{S(T)\}.\]
	通常称$S$为$f(x)$在$[a,b]$上的上积分,$s$为$f(x)$在$[a,b]$上的下积分.
	\begin{theorem}[达布定理]
		$$\lim_{\parallel T\parallel \rightarrow 0} s\left( T \right) =s,\quad \lim_{\parallel T\parallel \rightarrow 0} S\left( T \right) =S$$
	\end{theorem}
	\begin{theorem}[可积准则\uppercase\expandafter{\romannumeral1}]
		设函数$f(x)$在$[a,b]$上有界,则$f(x)$在$[a,b]$上可积的充分必要条件是:$f(x)$在$[a,b]$上的上积分等于下积分,即$S=s$.
	\end{theorem}
	\begin{theorem}[可积准则\uppercase\expandafter{\romannumeral2}]
		设函数$f(x)$在$[a,b]$上有界,则$f(x)$在$[a,b]$上可积的充要条件是:$\forall\varepsilon>0$,总存在一个分割$T$,使得
		\[S(T)-s(T)<\epsilon\]
		或
		\[\sum_{i=1}^{n}\omega_i\Delta x_i<\epsilon,\]
		其中$\omega_i(f)=M_i-m_i$称为$f(x)$在$\Delta_i=[x_{i-1},x_i]$上的振幅.
	\end{theorem}
	\begin{lemma}
		设$f(x)$在区间$\Delta$上有界,其振幅为$\omega(f)$,则
		$$\omega \left( f \right) =\mathop {\mathrm{sup}} \limits_{x^{\prime},x^{\prime \prime}\in \Delta}\left\{ |f\left( x^{\prime} \right) -f\left( x^{\prime \prime} \right) | \right\} .$$
	\end{lemma}
	\begin{theorem}[可积准则\uppercase\expandafter{\romannumeral3}]
		设函数$f(x)$在$[a,b]$上有界,则$f(x)$在$[a,b]$上可积的充分必要条件是:任给正数$\varepsilon,\eta$,总存在一个分割$T$,使得属于$T$的所有小区间中,对应于振幅$\omega_{k^\prime}\geqslant\epsilon$的那些小区间的总长$\dis\sum_{k^\prime}\Delta x_{k^\prime}<\eta$.
	\end{theorem}
	\begin{theorem}
		设$f(x)$在$[a,b]$上连续,则$f(x)$在$[a,b]$上可积.
	\end{theorem}
	\begin{theorem}
		设$f(x)$在$[a,b]$上单调,则$f(x)$在$[a,b]$上可积.
	\end{theorem}
	\begin{theorem}
		设$f(x)$是$[a,b]$上只有有限个间断点的有界函数,则$f(x)$在$[a,b]$上可积.
	\end{theorem}
	\begin{corollary}
		若$f(x)$是$[a,b]$上的分段连续函数(即只有有限个间断点,且都是第一类间断点的函数),则$f(x)$在$[a,b]$上可积.
	\end{corollary}
	\subsection{积分中值定理}
	\begin{theorem}[积分第一中值定理]
		设$f(x)$在$[a,b]$上连续,则至少存在一点$\xi\in(a,b)$,使得
		\[\int_a^bf(x)\dif x=f(\xi)(b-a).\]
	\end{theorem}
	\begin{theorem}[推广的第一积分中值定理]
		设$f(x)$在$[a,b]$上连续,$g(x)$在$[a,b]$上可积且不变号,则至少存在一点$\xi\in(a,b)$,使
		\[\int_a^bf(x)g(x)\dif x=f(\xi)\int_a^bg(x)\dif x.\]
	\end{theorem}
	\begin{theorem}[积分第二中值定理]
		设函数$f(x)$在$[a,b]$上可积,$g(x)$在$[a,b]$上递增且$g(x)\geqslant0$,则存在$\xi\in[a,b]$,使得
		\[\int_a^bf(x)g(x)\dif x=g(b)\int_\xi^bf(x)\dif x.\]
	\end{theorem}
	\begin{corollary}
		设函数$f(x)$在$[a,b]$上可积,$g(x)$在$[a,b]$上递减且$g(x)\geqslant0$,则存在$\xi\in[a,b]$,使得
		\[\int_a^bf(x)g(x)\dif x=g(a)\int_a^\xi f(x)\dif x.\]
	\end{corollary}
	\begin{theorem}[推广的积分第二中值定理]
		设函数$f(x)$在$[a,b]$上可积,$g(x)$为单调函数,则存在$\xi\in[a,b]$,使得
		\[\int_a^bf(x)g(x)\dif x=g(a)\int_a^\xi f(x)\dif x+g(b)\int_\xi^bf(x)\dif x\]
	\end{theorem}
	\subsection{关于定积分的一些补充}
	\begin{proposition}[Reimann引理]
		若$f(x)$在$[a,b]$上可积,$g(x)$以$T$为周期,在$[0,T]$上可积,则
		\[\lim_{n\to\infty}\int_a^bf(x)g(nx)\dif x=\dfrac{1}{T}\int_0^Tg(x)\dif x\int_a^bf(x)\dif x.\]
	\end{proposition}
\section{定积分的应用与反常积分}
	 \subsection{定积分应用的两种常用格式}
	 无.
	 \subsection{平面图形的面积}
	 \begin{theorem}[参数方程的情形]
	 	当曲线$C$由参数方程
	 	\[x=\varphi (t),y=\psi (t),t\in[\alpha,\beta]\]
	 	给出,在$[\alpha,\beta]$上$\psi (t)$连续,$\varphi (t)$连续可微,$x=\varphi (t)$在$[\alpha,\beta]$上严格递增或严格递减,则$C$与$x$轴及两条直线$x=\varphi (\alpha),x=\varphi (\beta)$围成的面积为
	 	\[A=\int_\alpha^\beta|\psi(t)\varphi^\prime(t)|\dif t.\]
	 \end{theorem}
	 \begin{theorem}[极坐标的情形]
	 	设由曲线$r=\varphi(\theta)$及射线$\theta=\alpha,\theta=\beta$围成一平面图形,其中$\varphi(\theta)\geqslant0$在$[\alpha,\beta]$上连续,$\beta-\alpha\leqslant 2\pi$,则它的面积为
	 	\[A=\frac12 \int_\alpha^\beta \varphi^2(\theta)\dif \theta\].
	 \end{theorem}
	 \subsection{平行截面面积求体积}
	 无.
	 \subsection{平面弧长的概念}
	 \begin{theorem}
	 	设简单曲线$C$的参数方程为
	 	 \[x=x(t),y=y(t),\quad t\in[\alpha,\beta]\]
	 	 且$C$为一光滑曲线,则$C$是可求长的且弧长为
	 	 \[L=\int_\alpha^\beta\sqrt{x^{\prime2}(t)+y^{\prime2}(t)}\dif t.\]
	 \end{theorem}
	 关于光滑曲线的定义可参见定义\ref{1}.
	 \begin{corollary}
	 	若简单曲线$C$由直角坐标方程
	 	\[y=f(x),\quad x\in[a,b]\]
	 	给出,则当$f$在$[a,b]$上连续可微时,此曲线为一光滑曲线.这时弧长公式为
	 	\[L=\int_a^b\sqrt{1+f^{\prime2}(x)}\dif x.\]
	 \end{corollary}
	 \begin{corollary}
	 	当简单曲线$C$由极坐标方程
	 	\[r=r(\theta),\quad \theta\in[\alpha,\beta]\]
	 	表示,$r(\theta)$在$[\alpha,\beta]$上连续可微且$r^2(\theta)+r^{\prime2}(\theta)\ne0$时,曲线弧长公式为
	 	\[L=\int_\alpha^\beta\sqrt{r^2(\theta)+r^{\prime2}(\theta)}\dif \theta.\]
	 \end{corollary}
	 \subsection{旋转曲面的面积}
	 	待作者补充.
	 \subsection{定积分在某些物理问题中的应用}
	 无.
	 \subsection{反常积分的概念与基本性质}
	 \begin{example}
	 	反常积分$\displaystyle\int_{1}^{+\infty}\dfrac{\dif x}{x^p}$在$p>1$时收敛,在$p\leqslant1$时发散.
	 \end{example}
	 \subsection{反常积分的敛散性}
	 \begin{theorem}[反常积分的Cauchy收敛准则]
	 	如果$f$在任一$[a,u]\subset[a,\omega)$上可积,$\displaystyle\int_a^\omega f(x)\dif x$为反常积分,则$\displaystyle\int_a^\omega f(x)\dif x$收敛的充要条件时:对任意给定的$\varepsilon>0$,存在$B\in[a,\omega)$,使对一切$u_1,u_2\in[B,\omega)$有
	 	\[\left|\int_{u_1}^{u_2}f(x)\dif x\right|<\varepsilon.\]
	 \end{theorem}
	 \begin{theorem}
	 	绝对收敛的反常积分必收敛.
	 \end{theorem}
	 \begin{theorem}
	 	如果$f$在任一$[a,u]\subset[a,\omega)$上非负可积,$\displaystyle\int_a^\omega f(x)\dif x$为反常积分,则反常积分$\displaystyle\int_a^\omega f(x)\dif x$收敛的充分必要条件使$F(u)=\displaystyle\int_a^uf(x)\dif x$在$[a,\omega)$上有上界.
	 \end{theorem}
	 \begin{theorem}
	 	设$f,g$在任一$[a,u]\subset[a,\omega]$上可积,$\displaystyle\int_a^\omega f(x)\dif x$及$\displaystyle\int_a^\omega g(x)\dif x$为反常积分,在$[a,\omega)$上有$0\leqslant f(x)\leqslant g(x)$,则
	 	\begin{enumerate}[(1)]
	 		\item 当$\displaystyle\int_a^\omega g(x)\dif x$收敛时有$\displaystyle \int_a^\omega f(x)\dif x$收敛;
	 		\item 当$\displaystyle \int_a^\omega f(x)\dif x$发散时有$\displaystyle\int_a^\omega g(x)\dif x$发散.
	 	\end{enumerate}
	 \end{theorem}
	 \begin{corollary}[比较判别法的极限形式]
	 	设$f,g$在任一$[a,u]\subset[a,\omega)$上非负可积,$\displaystyle\int_a^\omega f(x)\dif x$及$\displaystyle\int_a^\omega g(x)\dif x$为反常积分.如果$\displaystyle \lim_{x\rightarrow\omega^-}\frac{f(x)}{g(x)}=c$,则有
	 	\begin{enumerate}[(1)]
	 		\item 当$0<c<+\infty$时,$\displaystyle\int_a^\omega f(x)\dif x$与$\displaystyle\int_a^\omega g(x)\dif x$同敛态;
	 		\item 当$c=0$时,由$\displaystyle\int_a^\omega g(x)\dif x$收敛可推知$\displaystyle\int_a^\omega f(x)\dif x$收敛;
	 		\item 当$c=+\infty$时,由$\displaystyle\int_a^\omega g(x)\dif x$发散可推知$\displaystyle\int_a^\omega f(x)\dif x$发散.
	 	\end{enumerate}
	 \end{corollary}
	 \begin{corollary}[Cauchy判别法]
	 	设$f$在$[a,+\infty)(a>0)$的任一闭子区间上可积.
	 	\begin{enumerate}[(1)]
	 		\item 若存在$p>1$和$C>0$,使得$|f(x)|\leqslant \dfrac{C}{x^p},x\in[a,+\infty)$,则$\displaystyle\int_a^{+\infty} |f(x)|\dif x$收敛;
	 		\item 若存在$p\leqslant1$和$C>0$,使得$|f(x)|\geqslant \dfrac{C}{x^p},x\in[a,+\infty)$,则$\displaystyle\int_a^{+\infty} |f(x)|\dif x$发散.
	 	\end{enumerate}
	 \end{corollary}
	 \begin{corollary}[Cauchy判别法]
	 	设$f$在$[a,+\infty)(a>0)$的任一闭子区间上可积,且存在实数$p$,使得
	 	\[\lim_{x\rightarrow+\infty}x^p|f(x)|=c,\]
	 	则
	 	\begin{enumerate}[(1)]
	 		\item 当$0\leqslant c<+\infty$且$p>1$时,$\displaystyle\int_a^{+\infty} |f(x)|\dif x$收敛;
	 		\item 当$0<c\leqslant+\infty$且$p\leqslant 1$时,$\displaystyle\int_a^{+\infty} |f(x)|\dif x$发散.
	 	\end{enumerate}
	 \end{corollary}
	 \begin{corollary}[Cauchy判别法]
	 	设$f$在$[a,b)$的任一闭子区间上可积且$b$为瑕点.
	 	\begin{enumerate}[(1)]
	 		\item 若存在$0<p<1$和$C>0$,使$|f(x)|\leqslant \dfrac{C}{(b-x)^p},x\in[a,b)$,则瑕积分$\displaystyle\int_a^b|f(x)|\dif x$收敛;
	 		\item 若存在$p\geqslant1$和$C>0$,使得$|f(x)|\geqslant \dfrac{C}{(b-x)^p},x\in[a,b)$,则瑕积分$\displaystyle\int_a^b|f(x)|\dif x$发散.
	 	\end{enumerate}
	 \end{corollary}
	 \begin{corollary}[Cauchy判别法]
	 	设$f$在$[a,b)$的任一闭子区间上可积,$b$为瑕点,且存在实数$p$,使得
	 	\[\lim_{x\rightarrow b^-}(b-x)^p|f(x)|=c,\]
	 	则
	 	\begin{enumerate}[(1)]
	 		\item 当$0\leqslant c<+\infty$且$0<p<1$时,瑕积分$\displaystyle\int_a^b|f(x)|\dif x$收敛;
	 		\item 当$0<c\leqslant+\infty$且$p\geqslant1$时,瑕积分$\displaystyle\int_a^b|f(x)|\dif x$发散.
	 	\end{enumerate}
	 \end{corollary}
	 \begin{theorem}[Dirichlet判别法]
	 	设$f(x),g(x)$在任一$[a,u]\subset[a,\omega)$上可积,$\displaystyle \int_a^\omega f(x)g(x)\dif x$是反常积分.若
	 	\begin{enumerate}[(1)]
	 		\item $F(u)=\displaystyle \int_a^u f(x)\dif x$在$[a,\omega)$上有界;
	 		\item $g(x)$在$[a,\omega)$上单调且$\displaystyle\lim_{x\rightarrow\omega^-}g(x)=0$,
	 	\end{enumerate}
	 	则反常积分$\displaystyle \int_a^\omega f(x)g(x)\dif x$收敛.
	 \end{theorem}
	 \begin{theorem}[Abel判别法]
	 	设$f,g$在任一$[a,u]\subset[a,\omega)$上可积,$\displaystyle \int_a^\omega f(x)g(x)\dif x$是反常积分.若$\displaystyle \int_a^\omega f(x)\dif x$收敛,$g(x)$在$[a,\omega)$上单调有界,则$\displaystyle \int_a^\omega f(x)g(x)\dif x$收敛.
	 \end{theorem}
\section{数项级数}
	\subsection{数项级数的概念与性质}
	暂无.
	\subsection{正项级数}
	\begin{theorem}
		设$a_n\geqslant0(n=1,2,\cdots)$,则级数$\displaystyle\sum_{n=1}^{\infty}a_n$收敛的充要条件是:它的部分和数列$\{s_n\}$有上界.
	\end{theorem}
	\begin{theorem}[Cauchy积分判别法]
		设函数$f$在$[1,+\infty)$上单调递减,并且非负,则级数$\displaystyle\sum_{n=1}^{\infty}f(n)$与无穷积分$\displaystyle \int_1^{+\infty}f(x)\dif x$具有相同的敛散性.
	\end{theorem}
	\begin{theorem}[比较判别法]
		设$\displaystyle\displaystyle\sum_{n=1}^{\infty}a_n$和$\displaystyle\displaystyle\sum_{n=1}^{\infty}b_n$是两个级数.若存在正整数$N$,使当$n\geqslant N$时有$0\leqslant a_n\leqslant b_n$,则
		\begin{enumerate}[(1)]
			\item 当级数$\displaystyle\displaystyle\sum_{n=1}^{\infty}b_n$收敛时,级数$\displaystyle\displaystyle\sum_{n=1}^{\infty}a_n$收敛;
			\item 当级数$\displaystyle\displaystyle\sum_{n=1}^{\infty}a_n$发散时,级数$\displaystyle\displaystyle\sum_{n=1}^{\infty}b_n$发散.
		\end{enumerate}
	\end{theorem}
	\begin{theorem}[比较判别法的极限形式]
		设有正项级数$\displaystyle\displaystyle\sum_{n=1}^{\infty}a_n$和$\displaystyle\displaystyle\sum_{n=1}^{\infty}b_n$.如果$\displaystyle\lim_{n\rightarrow \infty}\frac{a_n}{b_n}=l$,则
		\begin{enumerate}[(1)]
			\item 当$0<l<+\infty$时,级数$\displaystyle\displaystyle\sum_{n=1}^{\infty}a_n$与级数$\displaystyle\displaystyle\sum_{n=1}^{\infty}b_n$同敛散;
			\item 当$l=0$且级数$\displaystyle\displaystyle\sum_{n=1}^{\infty}b_n$收敛时.级数$\displaystyle\displaystyle\sum_{n=1}^{\infty}a_n$收敛;
			\item 当$l=+\infty$且级数$\displaystyle\displaystyle\sum_{n=1}^{\infty}b_n$发散时,级数$\displaystyle\displaystyle\sum_{n=1}^{\infty}a_n$发散.
		\end{enumerate}
	\end{theorem}
	\begin{theorem}[Cauchy根值法]
		设$\displaystyle\displaystyle\sum_{n=1}^{\infty}a_n$为正项级数.
		\begin{enumerate}[(1)]
			\item 如果存在$q\in(0,1),N\in\mathbb{N}_+$,使得
			\[\sqrt[n]{a_n}\leqslant q<1,\quad n>N,\]
			则级数$\displaystyle\displaystyle\sum_{n=1}^{\infty}a_n$收敛;
			\item 如果$\sqrt[n]{a_n}\geqslant1$对无穷多个$n$成立,则级数$\displaystyle\displaystyle\sum_{n=1}^{\infty}a_n$发散.
		\end{enumerate}
	\end{theorem}
	\begin{corollary}[Cauchy根植法的极限形式]
		设$\displaystyle\displaystyle\sum_{n=1}^{\infty}a_n$是正项级数,如果
		\[\lim_{n\rightarrow\infty}\sqrt[n]{a_n}=r,\]
		则
		\begin{enumerate}[(1)]
			\item 当$r<1$时,级数$\displaystyle\displaystyle\sum_{n=1}^{\infty}a_n$收敛;
			\item 当$r>1$时,级数$\displaystyle\displaystyle\sum_{n=1}^{\infty}a_n$发散.
		\end{enumerate}
	\end{corollary}
	\begin{theorem}[D'Alembert比值法]
		设$\displaystyle\sum_{n=1}^{\infty}a_n$为正项级数.
		\begin{enumerate}[(1)]
			\item 如果存在$q\in(0,1),N\in\mathbb{N}_+$,使得
			\[\frac{a_{n+1}}{a_n}\leqslant q,\quad n>N,\]
			则级数$\displaystyle\sum_{n=1}^{\infty}a_n$收敛;
			\item 如果存在$N\in\mathbb{N}_+$,使得$\dfrac{a_{n+1}}{a_n}\geqslant 1$对$n>N$成立,则级数$\displaystyle\sum_{n=1}^{\infty}a_n$发散.
		\end{enumerate}
	\end{theorem}
	\begin{corollary}[D'Alembert比值法的极限形式]
		设$\displaystyle\sum_{n=1}^{\infty}a_n$是正项级数,如果
		\[\lim_{n\rightarrow\infty}\frac{a_{n+1}}{a_n}=s,\]
		则
		\begin{enumerate}[(1)]
			\item 当$s<1$时,级数$\displaystyle\sum_{n=1}^{\infty}a_n$收敛;
			\item 当$s>1$时,级数$\displaystyle\sum_{n=1}^{\infty}a_n$发散.
		\end{enumerate}
	\end{corollary}
	\begin{theorem}[Cauchy根值法]
		设$\displaystyle\sum_{n=1}^{\infty}a_n$为正项级数.如果
		\[\varlimsup_{n\to\infty}\sqrt[n]{a_n}=r,\]
		则
		\begin{enumerate}[(1)]
			\item 当$r<1$时,级数$\displaystyle\sum_{n=1}^{\infty}a_n$收敛;
			\item 当$r>1$时,级数$\displaystyle\sum_{n=1}^{\infty}a_n$发散.
		\end{enumerate}
	\end{theorem}
	\begin{theorem}[D'Alembert比值法]
		设$\displaystyle\sum_{n=1}^{\infty}a_n$是正项级数.
		\begin{enumerate}[(1)]
			\item 若$\displaystyle\varlimsup_{n\to\infty}\frac{a_{n+1}}{a_n}=s<1$,则级数$\displaystyle\sum_{n=1}^{\infty}a_n$收敛;
			\item 若$\displaystyle \varliminf_{n\to\infty}\frac{a_{n+1}}{a_n}=s^\prime>1$,则级数$\displaystyle\sum_{n=1}^{\infty}a_n$发散.
		\end{enumerate}
	\end{theorem}
	\begin{theorem}[Raabe判别法]
		设$\displaystyle\sum_{n=1}^\infty a_n$是正项级数,满足
		\[\lim_{n\to \infty}\left(\frac{a_n}{a_{n+1}}-1\right)=r,\]
		则
		\begin{enumerate}[(1)]
			\item 当$r>1$时,级数$\displaystyle\sum_{n=1}^{\infty}a_n$收敛;
			\item 当$r<1$时,级数$\displaystyle\sum_{n=1}^{\infty}a_n$发散.
		\end{enumerate}
	\end{theorem}
	\subsection{一般项级数}
	\begin{theorem}
		若级数$\displaystyle\sum_{n=1}^{\infty}a_n$绝对收敛,则它必收敛.
	\end{theorem}
	\begin{theorem}[Leibniz判别法]
		Leibniz级数必收敛.
	\end{theorem}
	\begin{theorem}[Dirichlet判别法]
		如果数列$\{a_n\}$单调趋于0,而级数$\displaystyle\sum_{n=1}^{\infty}b_n$的部分和数列有界,则$\displaystyle \sum_{n=1}^{\infty}a_nb_n$收敛.
	\end{theorem}
	\begin{theorem}[Abel判别法]
		如果数列$\{a_n\}$单调有界,而级数$\displaystyle\sum_{n=1}^{\infty}b_n$收敛,则$\displaystyle \sum_{n=1}^{\infty}a_nb_n$收敛.
	\end{theorem}
	\subsection{绝对收敛与条件收敛级数的性质}
	作者暂时还没学.
\section{函数项级数}
	\subsection{函数列一致收敛的概念与判定}
	\begin{theorem}[余项定理]
		设函数列$\{f_n(x)\}$在集合$I$上逐点收敛与函数$f(x)$.引入记号
		\[d(f_n,f)=\sup_{x\in I}|f_n(x)-f(x)|,\]
		则下列三条是等价的.
		\begin{enumerate}[(1)]
			\item $\{f_n(x)\}$在集合$I$上一致收敛于函数$f(x)$;
			\item $\lim\limits_{n\to \infty}d(f_n,f)=0$;
			\item 对任何数列$\{x_n\}\subset I$都有$\lim\limits_{n\to \infty}|f_n(x_n)-f(x_n)|=0$.
		\end{enumerate}
	\end{theorem}
	\begin{theorem}[一致收敛的柯西准则]
		设函数列$\{f_n(x)\}$在集合$I$上有定义,则这函数列在$I$上一致收敛的充要条件是:对任意给定的$\varepsilon>0$,存在$N=N(\varepsilon)\in \mathbb{N}_+$,使当$n,m>N$时都有
		\[|f_m(x)-f_n(x)|<\varepsilon,\quad \forall x\in I.\]
	\end{theorem}
	\subsection{一致收敛函数列的性质}
	\begin{theorem}[连续性]
		若函数列$\{f_n(x)\}$的每一项$f_n(x)$在$[a,b]$上连续,并且$\{f_n(x)\}$在$[a,b]$上一致收敛于$f(x)$,则$f(x)$在$[a,b]$上连续.
	\end{theorem}
	\begin{theorem}[Dini定理]
		设函数列$\{f_n(x)\}$在$[a,b]$上逐点收敛于$f(x)$.如果
		\begin{enumerate}[(1)]
			\item $f_n(x)$都在$[a,b]$上连续,$n=1,2,\cdots;$
			\item 对任意固定的$x\in [a,b]$,数列$\{f_n(x)\}$对$n$都是单调递增的;
			\item 极限函数$f(x)$在$[a,b]$上连续,
		\end{enumerate}
		则函数列$\{f_n(x)\}$在闭区间$[a,b]$上一致收敛于$f(x)$.
	\end{theorem}
	\begin{theorem}[可积性]
		若函数列$\{f_n(x)\}$的每一项在$[a,b]$上连续,并且$\{f_n(x)\}$在$[a,b]$上一致收敛于$f_(x)$,则$f(x)$在$[a,b]$上可积且
		\[\lim_{n\to \infty}\int_a^bf_n(x)\dif x=\int_a^bf(x)\dif x.\]
	\end{theorem}
	\begin{theorem}[可导性]
		如果函数列$\{f_n(x)\}$满足条件
		\begin{enumerate}[(1)]
			\item $f_n(x)$在$[a,b]$上连续可微,$n=1,2,\cdots$;
			\item 导函数列$\{f_n^\prime(x)\}$在$[a,b]$上一致收敛于$\varphi(x)$;
			\item $\{f_n(x)\}$至少在某个$x_0\in [a,b]$收敛,即$\lim\limits_{n\to \infty}f_n(x_0)=y_0$,
		\end{enumerate}
		那么函数列$\{f_n(x)\}$在$[a,b]$上一致收敛于某个连续可微函数$f(x)$,并且$f^\prime(x)=\varphi(x)$.
	\end{theorem}
	\subsection{函数项级数一致收敛的概念及判定}
	\begin{theorem}[一致收敛的Cauchy准则]
		函数项级数$\sum\limits_{n=1}^\infty u_n(x)$在数集$E$上一致收敛的充分必要条件时:对任意给定的$\varepsilon>0$,存在$N\in \mathbb{N}_+$,使当$n>N$时,对一切$p\in \mathbb{N}_+$都有
		\[|u_{n+1}(x)+u_{n+2}(x)+\cdots+u_{n+p}(x)|<\varepsilon,\quad \forall x\in E.\]
	\end{theorem}
	\begin{corollary}
		函数项级数$\sum\limits_{n=1}^\infty u_n(x)$在数集$E$上一致收敛的必要条件是函数列$\{u_n(x)\}$在$E$上一致收敛于0.
	\end{corollary}
	\begin{definition}
		设函数项级数$\sum\limits_{n=1}^\infty u_n(x)$在$E$上有定义,称$\sum\limits_{k=n+1}^\infty u_k(x)$为其余级数.当$\sum\limits_{n=1}^\infty u_n(x)$在$E$上收敛时,称$\sum\limits_{k=n+1}^\infty u_k(x)$为其余项,记作$R_n(x)$,即
		\[R_n(x)=\sum_{k=n+1}^{\infty}u_k(x),\quad x\in E.\]
	\end{definition}
	\begin{theorem}[余项定理]
		设函数项级数$\sum\limits_{n=1}^\infty u_n(x)$在数集$E$上收敛,则$\sum\limits_{n=1}^\infty u_n(x)$在$E$上一致收敛的充分必要条件为
		\[\lim_{n\to\infty}\sup_{x\in E}|R_n(x)|=0.\]
	\end{theorem}
	\begin{theorem}[M判别法]
		设$\displaystyle\sum_{n=1}^\infty u_n(x)$是定义在数集$E$上的函数项级数.若存在收敛的常数项级数$\displaystyle\sum_{n=1}^\infty M_n$和$N_0\in \mathbb{N}_+$,使当$n\geqslant N_0$时,
		\[|u_n(x)|\leqslant M_n,\quad \forall x\in E,\]
		则函数项级数$\displaystyle\sum_{n=1}^\infty u_n(x)$在$E$上一致收敛.
	\end{theorem}
	\subsection{和函数的分析性质}
	\begin{theorem}[连续性]
		若$u_n(x) \ (n=1,2,\cdots)$都在区间$[a,b]$上连续,函数项级数$\dis\sum_{n=1}^{\infty}u_n(x)$在$[a,b]$上一致收敛于$S(x)$,则和函数$S(x)$在$[a,n]$上连续.
	\end{theorem}
	\begin{theorem}[Dini定理]
		设函数项级数$\dis\sum_{n=1}^{\infty}u_n(x)$在闭区间$[a,b]$上逐点收敛到$S(x)$且满足
		\begin{enumerate}[(1)]
			\item $u_n(x) \ (n=1,2,\cdots)$都在$[a,b]$上连续且非负;
			\item $S(x)$在$[a,b]$连续,
		\end{enumerate}
		则级数$\dis\sum_{n=1}^{\infty}u_n(x)$在$[a,b]$上一致收敛.
	\end{theorem}
	\begin{theorem}[逐项积分]
		若$u_n(x) \ (n=1,2,\cdots)$都在区间$[a,b]$上连续,函数项级数$\dis\sum_{n=1}^{\infty}u_n(x)$在$[a,b]$上一致收敛于$S(x)$,则$S(x)$在$[a,b]$上可积且
		\[\int_a^b\sum_{n=1}^\infty u_n(x)\dif x=\sum_{n=1}^\infty\int_a^bu_n(x)\dif x.\]
	\end{theorem}
	\begin{theorem}[逐项求导]
		若函数项级数$\dis\sum_{n=1}^{\infty}u_n(x)$满足
		\begin{enumerate}[(1)]
			\item $u_n(x) \ (n=1,2,\cdots)$在$[a,b]$上连续可导;
			\item $\dis\sum_{n=1}^{\infty}u_n(x)$在某点$x_0\in [a,b]$收敛;
			\item $\dis\sum_{n=1}^{\infty}u_n^\prime(x)$在$[a,b]$上一致收敛到$\sigma(x)$,
		\end{enumerate}
		则$\dis\sum_{n=1}^{\infty}u_n(x)$在$[a,b]$上一致收敛到某个连续可导函数$S(x)$且$S^\prime(x)=\sigma(x)$,即
		\[\left(\sum_{n=1}^{\infty}u_n(x)\right)^\prime=\sum_{n=1}^{\infty}u_n^\prime(x)\]
	\end{theorem}
	\subsection{处处不可微的连续函数}
	作者暂时还没学.
\section{幂级数与Fourier级数}
	\subsection{幂级数的收敛域与和函数}
	\begin{theorem}[Abel第一定理]
		\begin{enumerate}[(1)]
			\item 若幂级数$\dis\sum_{n=0}^\infty a_nx^n$在$x_1(\not=0)$处收敛,则对于满足不等式$|x|<|x_1|$的一切$x$,幂级数$\dis\sum_{n=0}^\infty a_nx^n$都是绝对收敛;
			\item 若幂级数$\dis\sum_{n=0}^\infty a_nx^n$在$x_2(\not=0)$处发散,则对于满足不等式$|x|>|x_2|$的一切$x$,幂级数$\dis\sum_{n=0}^\infty a_nx^n$都发散.
		\end{enumerate}
	\end{theorem}
	\begin{theorem}
		设幂级数$\dis\sum_{n=0}^\infty a_nx^n$的收敛半径$R>0$,则
		\begin{enumerate}[(1)]
			\item 幂级数$\dis\sum_{n=0}^\infty a_nx^n$在区间$(-R,R)$内每一点绝对收敛;
			\item 幂级数$\dis\sum_{n=0}^\infty a_nx^n$在任意$x\not\in [-R,R]$处发散;
			\item 当$x=\pm R$时,幂级数$\dis\sum_{n=0}^\infty a_nx^n$可能收敛,也可能发散.
		\end{enumerate}
	\end{theorem}
	\begin{theorem}
		对于给定的级数$\dis\sum_{n=0}^\infty a_nx^n$,如果
		\[\lim_{n\to\infty}\left|\frac{a_{n+1}}{a_n}\right|=\rho\]
		或
		\[\lim_{n\to\infty}\sqrt[n]{|a_n|}=\rho,\]
		则其收敛半径$R=\dfrac1\rho\left(\text{这里约定}\rho=0\text{时},\text{定义}\dfrac1\rho=+\infty;\rho=+\infty\text{时,定义}\dfrac1\rho=0\right)$.
	\end{theorem}
	\begin{theorem}[Cauchy-Hadamand定理]
		对于给定的幂级数$\dis\sum_{n=0}^\infty a_nx^n$,如果
		\[\varlimsup_{n\to \infty}\sqrt[n]{\left|a_n\right|}=\rho,\]
		则其收敛半径$R=\dfrac1\rho\left(\text{同样地,约定当}\rho=0\text{时},\text{定义}\dfrac1\rho=+\infty;\rho=+\infty\text{时,定义}\dfrac1\rho=0\right)$.
	\end{theorem}
	\begin{theorem}[Abel第二定理]
		\begin{enumerate}[(1)]
			\item 若幂级数$\dis\sum_{n=0}^\infty a_nx^n$的收敛半径$R>0$,则对任意取定的$r\in(0,R)$,它在$[-r,r]$上一致收敛;
			\item 若幂级数$\dis\sum_{n=0}^\infty a_nx^n$的收敛半径$R>0$且它在$x=R$处收敛,则它在$[0,R]$上一致收敛;
			\item 若幂级数$\dis\sum_{n=0}^\infty a_nx^n$的收敛半径$R>0$且它在$x=-R$处收敛,则它在$[-R,0]$上一致收敛.
		\end{enumerate}
	\end{theorem}
	\begin{theorem}[连续性]
		若幂级数$\dis\sum_{n=0}^\infty a_nx^n$的收敛半径为$R>0$,则它的和函数$S(x)$在$(-R,R)$内连续.
	\end{theorem}
	\begin{corollary}
		若幂级数$\dis\sum_{n=0}^\infty a_nx^n$的收敛半径为$R>0$,并且它在$x=R$处收敛,则它的和函数$S(x)$在$[0,R]$上连续,特别地,
		\[\lim_{x\to R^-}\sum_{n=0}^\infty a_nx^n=\sum_{n=0}^{\infty}a_nR^n.\]
	\end{corollary}
	\begin{theorem}[逐项求导与逐项积分]
		若幂级数$\dis\sum_{n=0}^\infty a_nx^n$的收敛半径为$R>0$,和函数为$S(x)$,即
		\[S(x)=\sum_{n=0}^{\infty}a_nx^n=a_0+a_1x+a_2x^2+\cdots+a_nx^n+\cdots,-R<x<R,\]
		则幂级数在收敛区间内可以逐项求导与逐项积分,即
		\begin{equation}
			\begin{aligned}
				S^\prime(x)&=\left(\sum_{n=0}^{\infty}a_nx^n\right)^\prime=\sum_{n=1}^{\infty}na_nx^{n-1}\\&=a_1+2a_2x+\cdots+na_nx^{n-1}+\cdots,\quad -R<x<R
			\end{aligned}\label{2}
		\end{equation}
		和
		\begin{equation}
			\begin{aligned}
				\int_0^xS(t)\dif t&=\int_0^x\left(\sum_{n=0}^{\infty}a_nt^n\right)\dif t=\sum_{n=0}^{\infty}\frac{a_n}{n+1}x^{n+1}\\&=a_0x+\frac{a_1}{2}x^2+\cdots+\frac{a_n}{n+1}x^{n+1}+\cdots,\quad -R<x<R
			\end{aligned}\label{3}
		\end{equation}
		且\ref{2}式和\ref{3}式中的幂级数收敛半径仍然是$R$.
	\end{theorem}
	\subsection{函数的幂级数展开}
	几个常用的初等函数的幂级数展开:
	\begin{enumerate}
		\item $e^x$的展开式
		\[e^x=\sum_{n=0}^{\infty}\frac{x^n}{n!},\quad x\in \mathbb{R}.\]
		\item $\sin x$的展开式
		\[\sin x=\sum_{n=0}^{\infty}\frac{(-1)^n}{(2n+1)!}x^{2n+1},\quad x\in \mathbb{R}.\]
		\item $\cos x$的展开式
		\[\cos x=\sum_{n=0}^{\infty}\frac{(-1)^n}{(2n)!}x^{2n},\quad x\in \mathbb{R}.\]
		\item $(1+x)^\alpha(\alpha\in\mathbb{R})$的展开式
		\[(1+x)^\alpha=1+\alpha x+\frac{\alpha(\alpha-1)}{2!}x^2+\cdots+\frac{\alpha(\alpha-1)\cdots(\alpha-n+1)}{n!}x^n+\cdots,\quad x\in(-1,1).\]
		二项式展开的收敛情况如下:
		\begin{enumerate}[(1)]
			\item 当$\alpha\leqslant-1$时,收敛域为$(-1,1)$;
			\item 当$-1<\alpha<0$时,收敛域为$(-1,1]$;
			\item 当$\alpha>0$时,收敛域为$[-1,1]$.
		\end{enumerate}
		\item $\ln (1+x)$的展开式
		\[\ln (1+x)=\sum_{n=1}^{\infty}(-1)^{n-1}\frac{1}{n}x^n,\quad x\in(-1,1].\]
		\item $\arctan x$的展开式
		\[\arctan x=\sum_{n=0}^\infty(-1)^n\frac{1}{2n+1}x^{2n+1},\quad x\in [-1,1]\]
	\end{enumerate}
	\subsection{三角级数与Fourier级数}
	如果函数$f(x)$是以$2l$为周期的周期函数,并且在区间$[-l,l]$上可积,则$f(x)$的以$2l$为周期的Fourier级数为
	\[f(x)\sim\frac{a_0}{2}+\sum_{n=1}^{\infty}\left(a_n\cos \frac{n\pi x}{l}+b_n\sin \frac{n\pi x}{l}\right),\]
	其中
	\[a_n=\frac1l\int_{-l}^lf(x)\cos \frac{n\pi x}{l}\dif x,\quad n=0,1,2,\cdots,\]
	\[b_n=\frac1l\int_{-l}^lf(x)\sin \frac{n\pi x}{l}\dif x,\quad n=1,2,\cdots.\]
	\subsection{Fourier级数的收敛性}
	作者暂时还没学.
\section{多元函数及其微分学}
	\subsection{平面中的点集}
	\begin{definition}
		如果点集$E$中的每一点都是$E$的内点,则称$E$是开集.如果点集$E$的所有聚点都属于$E$,则称$E$是闭集.
	\end{definition}
	\subsection{\texorpdfstring{$\mathbb{R}^2$}的完备性}
	无.
	\subsection{二元函数的极限和连续性}
	无.
	\subsection{多元函数的偏导数和全微分}
	\begin{theorem}[可微的必要条件]
		设二元函数$f$在$P_0(x_0,y_0)$某邻域内有定义,在点$P_0(x_0,y_0)$可微,则函数$f$在点$P_0(x_0,y_0)$的全微分可以表示为
		\[\dif f|_{(x_0,y_0)}=f_x(x_0,y_0)\dif x+f_y(x_0,y_0)\dif y.\]
	\end{theorem}
	\begin{corollary}
		设$z=f(x,y)$在$P_0(x_0,y_0)$的某邻域内有定义,则它在$P_0(x_0,y_0)$处可微的充要条件是:两个偏导数$f_x(x_0,y_0),f_y(x_0,y_0)$都存在,且满足
		\[\lim_{\rho\to 0}\frac{\Delta f(x_0,y_0)-(f_x(x_0,y_0)\Delta x+f_y(x_0,y_0)\Delta y)}{\rho}=0.\]
	\end{corollary}
	\subsection{复合函数的微分法}
	\begin{theorem}
		设函数$z=f(x,y)$可微,$x=\varphi t$和$y=\psi t$可导,则复合函数$z=f(\varphi (t),\psi (t))$也可导,并且
		\[\frac{\dif z}{\dif t}=\frac{\partial z}{\partial x}\frac{\dif x}{\dif t}+\frac{\partial z}{\partial y}\frac{\dif y}{\dif t}=f_x(\varphi(t),\psi (t))\varphi^{\prime}(t)+f_y(\varphi(t),\psi (t))\psi^\prime(t).\]
	\end{theorem}
	\begin{theorem}[复合函数偏导链式法则]
		设函数$z=f(u,v)$可微,$u=\varphi(x,y)$和$v=\psi(x,y)$都存在偏导数,则复合函数$z=f(\varphi(x,y),\psi(x,y))$也存在偏导数,并且
		\[\frac{\partial z}{\partial x}=\frac{\partial z}{\partial u}\frac{\partial u}{\partial x}+\frac{\partial z}{\partial v}\frac{\partial v}{\partial x},\]
		\[\frac{\partial z}{\partial y}=\frac{\partial z}{\partial u}\frac{\partial u}{\partial y}+\frac{\partial z}{\partial v}\frac{\partial v}{\partial y}.\]
	\end{theorem}
	\begin{theorem}
		设函数$z=f(u,v)$在$(u,v)$可微,$u=\varphi(x,y)$和$v=\psi(x,y)$分别在$(x,y)$可微,则复合函数$z=f[\varphi(x,y),\psi(x,y)]$在$(x,y)$也可微,并且
		\[\dif z=\left(\frac{\partial f}{\partial u}\frac{\partial u}{\partial x}+\frac{\partial f}{\partial v}\frac{\partial v}{\partial x}\right)\dif x+\left(\frac{\partial f}{\partial u}\frac{\partial u}{\partial y}+\frac{\partial f}{\partial v}\frac{\partial v}{\partial y}\right)\dif y.\]
	\end{theorem}
	\begin{theorem}
		设函数$z=f(x,y)$的混合偏导数$f_{xy}(x,y)$和$f_{yx}(x,y)$都在点$(x_0,y_0)$存在,并且至少有一个在该点连续,则二者相等,
		\[f_{xy}(x_0,y_0)=f_{yx}(x_0,y_0).\]
	\end{theorem}
\section{多元函数微分法的应用}
	\subsection{方向导数}
	\begin{theorem}
		如果函数$f$在点$P_0(x_0,y_0)$处可微,则$f$在点$P_0$沿任何方向$\bm l$的方向导数都存在,并且
		\[f_{\bm l}(P_0)=f_x(P_0)\cos \alpha+f_y(P_0)\cos \beta,\]
		其中$\cos \alpha,\cos \beta$是方向$\bm l$的方向余弦.
	\end{theorem}
	\subsection{多元函数Taylor公式}
	\begin{theorem}[Lagrange型余项的Taylor公式]
	设二元函数$f(x,y)$在点$P_0(x_0,y_0)$的某个领域$U(P_0)$内直到$n+1$阶连续可微(即具有直到$n+1$阶的连续偏导数),则对于任意一点$P(x_0+h,y_0+k)\in U(P_0)$,存在相应的$\theta\in (0,1)$,使得
	\begin{equation}\label{14.2}
	\begin{aligned}
		f(x_0+h,y_0+k)=&f(x_0,y_0)+\left(h\frac{\partial}{\partial x}+k\frac{\partial}{\partial y}\right)f(x_0,y_0)+\frac{1}{2!}\left(h\frac{\partial}{\partial x}+k\frac{\partial}{\partial y}\right)^2f(x_0,y_0)\\
		&+\cdots+\frac{1}{n!}\left(h\frac{\partial}{\partial x}+k\frac{\partial}{\partial y}\right)^nf(x_0,y_0)\\
		&+\frac{1}{(n+1)!}\left(h\frac{\partial}{\partial x}+k\frac{\partial}{\partial y}\right)^{n+1}f(x_0+\theta h,y_0+\theta k),
	\end{aligned}
	\end{equation}
	其中每一项都含有形式上的符号运算,即
	\[\left(h\frac{\partial}{\partial x}+k\frac{\partial}{\partial y}\right)^mf(x_0,y_0)=\sum_{i=0}^{m}\mathrm{C}_m^i\left.\frac{\partial^mf(x,y)}{\partial x^i\partial y^{m-i}}\right|_{(x_0,y_0)}h^ik^{m-i}.\]
	式\ref{14.2}称为二元函数$f(x,y)$在点$P(x_0,y_0)$的$n$阶带Lagrange型余项的Taylor公式(当$(x_0,y_0)=(0,0)$时,也称为Maclaurin公式).
	\end{theorem}
	\subsection{多元函数的极值}
	为了方便叙述定理\ref{14.3},记
	\[A:=f_{xx}(P_0),\quad B:=f_{xy}(P_0),\quad C:=f_{yy}(P_0),\]
	\[H:=\det \bm{H}_f(P_0)=AC-B^2.\]
	\begin{theorem}[极值的充分条件]\label{14.3}
		设函数$z=f(x,y)$在点$P_0(x_0,y_0)$的某领域$U(P_0)$内具有二阶连续偏导数,又设$P_0$是$f$的一个稳定点,
		\begin{enumerate}
			\item 当$H>0$时,如果$A>0$(或$C>0$),则函数$f$在$P_0$处取得极小值;如果$A<0$(或$C<0$),则函数$f$在$P_0$处取得极大值;
			\item 当$H<0$,函数$f$在$P_0$处不能取得极值.
		\end{enumerate}
	\end{theorem}
	\subsection{隐函数}
	\begin{theorem}[隐函数的存在唯一性]
		设函数$z=F(x,y)$满足下列条件:
		\begin{enumerate}
			\item $F(x_0,y_0)=0$;
			\item $F$在以点$P_0(x_0,y_0)$为内点的某一区域$D\subset \mathbb{R}^2$中连续;
			\item $F$在$D$内关于$y$是严格单调的,
		\end{enumerate}
		则在点$P_0$的某邻域$U(P_0)\subset D$内,由方程$F(x,y)=0$可以唯一地确定一个定义在某区间$(x_0-\alpha,x_0+\alpha)$内的(隐)函数$y=f(x)$,使得
		\begin{enumerate}
			\item $f(x_0)=y_0,\{(x,f(x))|x\in (x_0-\alpha,x_0+\alpha)\}\subset U(P_0)$且
			\[F(x,f(x))\equiv0,\quad x\in(x_0-\alpha,x_0+\alpha);\]
			\item $y=f(x)$在$(x_0-\alpha,x_0+\alpha)$内连续.
		\end{enumerate}
	\end{theorem}
	\begin{theorem}[隐函数的可微性]
		设函数$z=F(x,y)$满足下列条件:
		\begin{enumerate}
			\item $F(x_0,y_0)=0$;
			\item $F$在以点$P_0(x_0,y_0)$为内点的某一区域$D\subset \mathbb{R}^2$中连续;
			\item $F$在$D$内存在连续的偏导数$F_y(x,y)$且$F_y(x_0,y_0)\not=0$,
		\end{enumerate}
		则在点$P_0$的某邻域$U(P_0)\subset D$内,由方程$F(x,y)=0$可以唯一地确定一个定义在某区间$(x_0-\alpha,x_0+\alpha)$内连续的(隐)函数$y=f(x)$,使得
		\begin{enumerate}
			\item $f(x_0)=y_0,\{(x,f(x))|x\in (x_0-\alpha,x_0+\alpha)\}\subset U(P_0)$且
			\[F(x,f(x))\equiv0,\quad x\in(x_0-\alpha,x_0+\alpha);\]
			\item 假设$F_x(x,y)$在$D$内存在且连续,则隐函数$y=f(x)$在$(x_0-\alpha,x_0+\alpha)$中有连续导函数,并且
			\[f^\prime(x)=-\frac{F_x(x,y)}{F_y(x,y)}.\]
		\end{enumerate}
	\end{theorem}
	\begin{theorem}
		设函数$u=F(x,y,z)$满足下列条件:
		\begin{enumerate}
			\item $F(x_0,y_0,z_0)=0$;
			\item $F$在以点$P_0(x_0,y_0,z_0)$为内点的某一区域$D\subset \mathbb{R}^3$中连续;
			\item $F$在$D$内存在连续的偏导数$F_x,F_y$和$F_z$,并且$F_z(x_0,y_0,z_0)\not=0$,
		\end{enumerate}
		则在点$P_0$的某邻域$U(P_0)\subset D$内,由方程$F(x,y,z)=0$可以唯一地确定一个定义在某二维区域$U(x_0,y_0)$内的二元(隐)函数$z=f(x,y)$,使得
		\begin{enumerate}
			\item $f(x_0,y_0)=z_0,\{(x,y,f(x,y))|(x,y)\in U(x_0,y_0)\subset U(P_0)\}$且
			\[F(x,y,f(x,y))\equiv0,\quad (x,y)\in U(x_0,y_0);\]
			\item $z=f(x,y)$在$U(x_0,y_0)$内具有连续的偏导数$z_x,z_y$且
			\[f_x(x,y)=-\frac{F_x}{F_z},\quad f_y(x,y)=-\frac{F_y}{F_z}.\]
		\end{enumerate}
	\end{theorem}
	\subsection{隐函数组}
	\begin{theorem}[隐函数组的存在唯一性]
	设函数组$F(x,y,z)$和$G(x,y,z)$满足下列条件:
	\begin{enumerate}
		\item $F(x_0,y_0,z_0)=0,G(x_0,y_0,z_0)=0$;
		\item $F$和$G$在以点$P_0(x_0,y_0,z_0)$为内点的区域$V\subset \mathbb{R}^3$存在一阶连续偏导数;
		\item $F,G$关于$y,z$的Jacobi行列式$\left.\frac{\partial(F,G)}{\partial(y,z)}\right|_{P_0}\not=0,$
	\end{enumerate}
	则在点$P_0$的某邻域$U(P_0)\subset V$内,由方程组
	\[
	\begin{cases}
		F(x,y,z)=0,\\
		G(x,y,z)=0,\\
	\end{cases}
	\quad
	(x,y,z)\in V,
	\]
	可以唯一地确定一个定义在点$x_0$的邻域$U(x_0)$内的一元(隐)函数组
	\[\begin{cases}
		y=\varphi(x),\\
		z=\psi(x),
	\end{cases}\]
	使得
	\begin{enumerate}
		\item $y_0=\varphi(x_0),z_0=\psi(x_0)$且$\{(x,\varphi(x),\psi(x))|x\in U(x_0)\}\subset U(P_0)$,进而有恒等式
		\[
		\begin{cases}
			F(x,\varphi(x),\psi(x))\equiv0,\\
			G(x,\varphi(x),\psi(x))\equiv0,
		\end{cases}
		\quad
		x\in U(x_0);
		\]
		\item $y=\varphi (x)$和$z=\psi(x)$在$U(x_0)$内连续;
		\item $y=\varphi (x)$和$z=\psi(x)$在$U(x_0)$内有一阶连续的导数$\varphi^\prime(x),\psi^\prime(x)$且
		\[\frac{\dif \varphi}{\dif x}=\frac1J\frac{\partial(F,G)}{\partial(z,x)},\quad \frac{\dif \psi}{\dif x}=\frac1J\frac{\partial(F,G)}{\partial(x,y)},\]
		其中
		\[
		J=\frac{\partial(F,G)}{\partial(y,z)},\quad \frac{\partial(F,G)}{\partial(x,y)}=
		\begin{vmatrix}
			F_x & F_y\\
			G_x & G_y\\
		\end{vmatrix},
		\quad
		\frac{\partial(F,G)}{\partial(z,x)}=
		\begin{vmatrix}
			F_z & F_x\\
			G_z & G_x
		\end{vmatrix}.
		\]
	\end{enumerate}
	\end{theorem}
	\begin{theorem}[反函数的存在唯一性]
		设函数组
		\[
		\begin{cases}
			u=u(x,y),\\
			v=v(x,y),
		\end{cases}
		\quad
		(x,y)\in D.
		\]
		中的函数都在$D$上有连续的一阶偏导数,点$P_0(x_0,y_0)$是$D$的内点.进一步设
		\[u_0=u(x_0,y_0),\quad v_0=v(x_0,y_0),\quad 
		\left.\frac{\partial(u,v)}{\partial(x,y)}\right|_{P_0}\not=0,\]
		则在点$P_0^\prime(u_0,v_0)$的某邻域$U(P_0^\prime)$内存在唯一的一组反函数$x=x(u,v),y=y(u,v)$,使得
		\[x_0=x(u_0,v_0),\quad y_0=y(u_0,v_0),\quad \{(x(u,v),y(u,v))|(u,v)\in U(P_0^\prime)\}\subset U(P_0).\]
		进而,恒等式
		\[
		\begin{cases}
			u\equiv u(x(u,v),y(u,v)),\\
			v\equiv v(x(u,v),y(u,v)),
		\end{cases}
		\]
		和
		\[
		\begin{cases}
			x\equiv x(u(x,y),v(x,y)),\\
			y\equiv y(u(x,y),v(x,y)),
		\end{cases}
		\]
		分别在$U(P_0^\prime)$和$P_0$的某邻域中成立,并且$x=x(u,v),y=y(u,v)$在$U(P_0^\prime)$内存在连续的一阶偏导数,
		\[\frac{\partial x}{\partial u}=\frac{\partial v}{\partial y}\bigg/\frac{\partial(u,v)}{\partial(x,y)},\quad 
		\frac{\partial x}{\partial v}=-\frac{\partial u}{\partial y}\bigg/\frac{\partial(u,v)}{\partial(x,y)},\]
		\[\frac{\partial y}{\partial u}=-\frac{\partial v}{\partial x}\bigg/\frac{\partial(u,v)}{\partial(x,y)},\quad 
		\frac{\partial y}{\partial v}=\frac{\partial u}{\partial x}\bigg/\frac{\partial(u,v)}{\partial(x,y)}\]
	\end{theorem}
\section{含参变量积分}
\subsection{含参变量正常积分及其分析性质}
\begin{theorem}[含参变量积分的连续性]
	设函数$f(x,t)$在矩形区域$R=[a,b]\times[c,d]$上连续,则含参变量正常积分
	\[I(t)=\int_a^b f(x,t)\dif x,\quad t\in[c,d]\]
	在$[c,d]$上也连续.
\end{theorem}
\begin{theorem}[含参变量积分的可微性]
	设函数$f(x,t)$和$f_t(x,t)$在矩形区域$R=[a,b]\times[c,d]$上连续,则含参变量积分
	\[I(t)=\int_a^b f(x,t)\dif x,\quad t\in[c,d]\]
	在$[c,d]$上也可微且
	\[\dfrac{\dif}{\dif t}I(t)=\int_a^b \dfrac{\partial}{\partial t}f(x,t)\dif x,\quad t\in[c,d].\]
\end{theorem}
\begin{theorem}[含参变量积分的可积性]
	设函数$f(x,t)$在矩形区域$R=[a,b]\times[c,d]$上连续,则含参变量积分
	\[I(t)=\int_a^b f(x,t)\dif x,\quad t\in[c,d]\]
	在$[c,d]$上可积,并且有下列等式成立:
	\[\int_c^d{I\left( t \right) \mathrm{d}t=\int_c^d{\mathrm{d}t\int_a^b{f\left( x,t \right) \mathrm{d}x=\int_a^b{\left\{ \int_c^d{f\left( x,t \right) \mathrm{d}t} \right\} \mathrm{d}x.}}}}\]
\end{theorem}
\begin{theorem}[含参变量积分的可微性]
	设函数$f(x,t)$和$f_t(x,t)$在矩形区域$R=[a,b]\times[c,d]$上连续,函数$\varphi(t)$和$\psi(t)$在$[c,d]$上可导且
	\[a\leqslant\varphi(t)\leqslant b,\quad a\leqslant\psi(t)\leqslant b,\quad t\in[c,d],\]
	则含参变量积分$\dis I\left( t \right) =\int_{\varphi (t)}^{\psi (t)}{f\left( x,t \right) \mathrm{d}x}$在$[c,d]$上可导且
	\[\frac{\mathrm{d}}{\mathrm{d}t}I\left( t \right) =\int_{\varphi (t)}^{\psi (t)}{\frac{\partial f\left( x,t \right)}{\partial t}\mathrm{d}x+f}\left( \psi (t),t \right) \psi ^{\prime}(t)-f\left( \varphi (t),t \right) \varphi ^{\prime}(t).\]
\end{theorem}
\subsection{含参变量反常积分及一致收敛判别法}
\begin{definition}
	设对于区间$J$中的每一点$t$,含参变量$t$的反常积分$\dis I(t)=\int_a^{+\infty}f(x,t)\dif x$都收敛.如果对于任意给定的$\varepsilon>0$,存在与$t$无关的$G\in(a,+\infty)$,使得对于任意的$G^\prime>G$和$t\in J$均成立
	$$\left| \int_a^{+\infty}{f\left( x,t \right) \mathrm{d}x}-\int_a^{G^{\prime}}{f\left( x,t \right) \mathrm{d}x} \right|=\left| \int_{G^{\prime}}^{+\infty}{f\left( x,t \right) \mathrm{d}x} \right|<\varepsilon ,$$
	则称含参变量反常积分$\dis I(t)=\int_a^{+\infty}f(x,t)\dif x$在区间$J$上一致收敛或关于$t\in J$一致收敛.
\end{definition}
\begin{theorem}[一致收敛的Cauchy准则]
	含参变量反常积分$\dis I(t)=\int_a^{+\infty}f(x,t)\dif x$在区间$J$上一致收敛的充分必要条件是:对于任意给定的$\varepsilon>0$,总存在$G>a$,使得只要$G<G_1,G_2<+\infty$,就成立
	\[\left|\int_{G_1}^{G_2}f(x,t)\dif x\right|<\varepsilon,\quad \forall t\in J.\]
\end{theorem}
\begin{theorem}[M判别法]
	如果含参变量反常积分$\dis I(t)=\int_a^{+\infty}f(x,t)\dif x$的被积函数$f(x,t)$可被一个可积的函数$F(x)$所控制,即存在$b\geqslant a$,使得
	\[|f(x,t)|\leqslant F(x),\quad \forall x\in[b,+\infty),\forall t\in J,\]
	并且反常积分$\dis\int_b^{+\infty}F(x)\dif x$收敛,则含参变量反常积分$\dis I(t)=\int_a^{+\infty}f(x,t)\dif x$在区间$J$上是(绝对)一致收敛的.
\end{theorem}
\begin{theorem}[一致收敛的Dirichlet判别法]
	设$f(x,t)$和$g(x,t)$在$[a,+\infty)\times J$上连续且满足如下条件:
	\begin{enumerate}[(1)]
		\item 存在正常数$M$,使得$\dis \left|\int_a^bf(x,t)\dif x\right|\leqslant M$对所有的$b$满足$a<b<+\infty$和$t\in J$;
		\item 函数$g(x,t)$是$x$的单调函数;
		\item 当$x\to +\infty$时,$g(x,t)$关于$t\in J$一致趋于0,
	\end{enumerate}
	则含参变量反常积分$\dis I(t)=\int_{a}^{+\infty}f(x,t)g(x,t)\dif x$在区间$J$上一致收敛.
\end{theorem}
\begin{theorem}[一致收敛的Abel判别法]
	设$f(x,t)$和$g(x,t)$在$[a,+\infty)\times J$上连续且满足如下条件:
	\begin{enumerate}[(1)]
		\item 含参变量反常积分$\dis \int_a^{+\infty}f(x,t)\dif x$在区间$J$上一致收敛;
		\item 函数$g(x,t)$是$x$的单调函数;
		\item 函数$g(x,t)$在区间$J$上一致有界,即存在$M>0$,使得
		\[|g(x,t)|\leqslant M,\quad (x,t)\in[a,+\infty)\times J,\]
	\end{enumerate}
	则含参变量反常积分$\dis I(t)=\int_{a}^{+\infty}f(x,t)g(x,t)\dif x$在区间$J$上一致收敛.
\end{theorem}
\subsection{含参变量反常积分的分析性质}
\begin{theorem}[连续性]
	设函数$f(x,t)$在矩形区域$R=[a,+\infty)\times J$上连续,其中$J$是一个区间.又含参变量反常积分$\dis I(t)=\int_a^{+\infty}f(x,t)\dif x$在区间$J$上是一致收敛的,则含参变量反常积分
	\[\dis I(t)=\int_a^{+\infty}f(x,t)\dif x\]
	在$J$上连续.
\end{theorem}
\begin{theorem}[可积性]
	设函数$f(x,t)$在矩形区域$R=[a,+\infty)\times J$上连续,并且含参变量反常积分$\dis I(t)=\int_a^{+\infty}f(x,t)\dif x$在区间$J$上一致收敛,则$I(t)$在任意有限区间$[c,d]\subset J$上可积且
	$$\int_c^d{I\left( t \right) \mathrm{d}t=\int_c^d{\left\{ \int_a^{+\infty}{f\left( x,t \right) \mathrm{d}x} \right\} \mathrm{d}t=\int_a^{+\infty}{\left\{ \int_c^d{f\left( x,t \right) \mathrm{d}t} \right\} \mathrm{d}x.}}}$$
\end{theorem}
\begin{theorem}[可微性]
	设函数$f(x,t)$和$f_t(x,t)$在矩形区域$R=[a,+\infty)\times J$上连续且$\dis I(t)=\int_a^{+\infty}f(x,t)\dif x$在$J$上收敛,$\dis \int_a^{+\infty}f_t(x,t)\dif x$在$J$上一致收敛,则含参变量反常积分$I(t)$在$J$上可导且
	\[\dfrac{\dif}{\dif t}I(t)=\int_a^{+\infty} \dfrac{\partial}{\partial t}f(x,t)\dif x,\quad t\in J.\]
\end{theorem}
\begin{theorem}[积分顺序的可交换性]
	设$f(x,y)$在$R=[a,+\infty)\times[c,+\infty)$上连续,含参变量反常积分$\dis\varphi(x)=\int_c^{+\infty}f(x,y)\dif y$和$\dis\psi(y)=\int_a^{+\infty}f(x,y)\dif x$分别在区间$[a,+\infty)$和$[c,+\infty)$中收敛,并且
	\begin{enumerate}[(1)]
		\item 积分$\varphi(x)$和$\psi(y)$在$[a,+\infty)$及$[c,+\infty)$上分别具有内闭一致收敛性,即$\varphi(x)$关于$x$在任何区间$[a,b]$上一致收敛,$\psi(y)$关于$y$在任何区间$[c,d]$上一致收敛;
		\item 下面的两个积分中至少有一个收敛:
		$$\int_a^{+\infty}{\mathrm{d}x}\int_c^{+\infty}{|f\left( x,y \right) |\mathrm{d}y},\quad \int_c^{+\infty}{\mathrm{d}y\int_a^{+\infty}{|f\left( x,y \right) |\mathrm{d}x}},$$
		则
		$$\int_a^{+\infty}{\mathrm{d}x\int_c^{+\infty}{f\left( x,y \right) \mathrm{d}y,\quad \int_c^{+\infty}{\mathrm{d}y\int_a^{+\infty}{f\left( x,y \right) \mathrm{d}x}}}}.$$
	\end{enumerate}
\end{theorem}
\end{document}





